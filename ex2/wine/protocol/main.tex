\documentclass{acm_proc_article-sp}

\usepackage{graphicx}% Include figure files
\usepackage{dcolumn}% Align table columns on decimal point
\usepackage{bm}% bold math
\usepackage{float}
\usepackage{array}
\usepackage{verbatim}
\usepackage{hyperref}% add hypertext capabilities
%\usepackage[mathlines]{lineno}% Enable numbering of text and display math
%\linenumbers\relax % Commence numbering lines

\usepackage{listings}
\usepackage{footnote}
\makesavenoteenv{table}
\makesavenoteenv{table*}
\makesavenoteenv{tabular}

\begin{document}

\title{Self organizing systems WS14 - Exercise 2\\
       Self Organizing Maps}% Force line breaks with \\

\numberofauthors{2}
\author{
\alignauthor
Dragan Avramovski\\
       \email{e1426093@student.tuwien.ac.at}
\alignauthor
Richard Plangger\\
 \email{e1025637@student.tuwien.ac.at}
}

\date{\today}

\maketitle


\begin{abstract}
\end{abstract}

\keywords{Self organizing systems, SOM }

\section{Wine dataset}

For our dataset we chose the ``wine-quality'' dataset from the UCI Machine Learning Repository~\cite{ucirepo}. 
Herefater the dataset is called WQ.
It is data from wine variants of a Portoguese wine called ``Vino Verde''.
The data points the quality measure are physicochemical values measured
by sensors or tests. There is no information about the grape types,
wine brand, selling prices of the wine.
WQ has 12 different attributes and more than 6500 instances of red and white wine listed in the following enumeration:

\begin{itemize}
    \item Fixed acidity
    \item Volatile acidity
    \item Citric acid
    \item Residual sugar
    \item Chlorides
    \item Free sulfur dioxide
    \item Total sulfur dioxide
    \item Density
    \item pH
    \item Sulphates
    \item Alcohol
    \item Quality (A score between 0 and 10)
    \item Wine type (Red or white wine)
\end{itemize}

The data set contains two seperate data files. One for white wine,
another for red wine. We merged the two together into a monolitic file,
and appending the wine type as a new attribute. $0$ denotes red wine, $1$ white wine.

The attribute quality is the only attribute that are human decided.
Each data entry has at least opinions of three different experts. This
sensory data is collected and the median the value included in the
dataset.

\section{Normalisation and data cleaning}

To get a feeling of the distribution of the data we used
WEKA to plot each attribute along its value range to visualise
density and the distribution.
Figure~\ref{fig:dist-alcohol} shows the distribution of the alcohol attribute. pH has
also a very similar distribution and both do not seem to have outliers.
For these two attributes we apply min max scaling.

\begin{figure}
\centering
\includegraphics[width=\linewidth]{img/dist-alcohol}
\caption{Data distribution of the alcohol attribute (not normalized)}
\label{fig:dist-alcohol}
\end{figure}

\begin{comment}
    \begin{figure}
    \centering
    \includegraphics[width=\linewidth]{img/dist-sulphates}
    \caption{Data distribution of the sulphate attribute (not normalized)}
    \label{fig:dist-sulphates}
    \end{figure}
\end{comment}


For Fixed/Volatile acidity and Total sulfur dioxide we apply Zero Mean Variance scaling. For these three we 
believe that the few outliers are features, not noise in the measurements.
For all others\footnote{Citri acid, Residual Sugar, Chlorides, Free/Total sulfur dioxide, Density, Sulphates}
we decided to exclude data samples with the outliers. The Table~\ref{tab:cutoff} shows the threshold after we drop
the data record. In Figure~\ref{fig:dist-citric-acid} a sample of the not normalized data is shown. A lot
of samples have a value to the left of the attribute range. Figure~\ref{fig:ndist-citric-acid} shows the
normalized attribute range.

\begin{figure}
\centering
\includegraphics[width=\linewidth]{img/dist-citric-acid}
\caption{Data distribution of the citric acid attribute (not normalized)}
\label{fig:dist-citric-acid}
\end{figure}

\begin{figure}
\centering
\includegraphics[width=\linewidth]{img/ndist-citric-acid}
\caption{Data distribution of the citric acid attribute (normalized)}
\label{fig:ndist-citric-acid}
\end{figure}

\begin{table}
\centering
\begin{tabular}{l|c}
    Attribute & Cutoff \\
    \hline
    \hline
    Citri acid & 1.0 \\
    \hline
    Residual Sugar & 30.0 \\
    \hline
    Chlorides & 0.3 \\
    \hline
    Free sulfur dioxide & 150 \\
    \hline
    Density & 1.01 \\
    \hline
    Sulphates & 1.3 \\
\end{tabular}
\caption{Table that shows the cutoff for normalizations}
\label{tab:cutoff}
\end{table}

The total amount of rows that are filtered are 44 rows. This decreases the data set from 6497 instances to 6453.
We provide the original data samples in a file called ``wq.csv'', the samples that do not include quality and wine type
in ``wq-n-o.csv'' and the normalized data in ``wq-n.csv''.

\begin{comment}
In our automated script we used Python 2.7. To repeat the experiment we set the random seed to 0.
It cannot occur that one data sample of the original samples is included twice in the subsampled dataset.
\end{comment}
We also include the Python script called ``wine.py'' that can be invoked\footnote{Example: python wine.py < wq.csv} on the original dataset ``wq.csv'' to generate all files needed by the SOMToolbox.

\subsection{Classification}

In the original data set the attribute quality were used for classification. As we merged the white and red wine,
we additionally binned the quality into 3 bins: Poor-, Average- and High- quality. We did this, to reduce the possible
amount of clusters (6 instead of approx. 14).
The data samples included in WQ covered a range from 3 to 9. Table~\ref{tab:quality-binning} shows the binning settings.
The bins are not equally distributed along the range, but we selected them by seeing the distribution in WEKA. Nearly
all bins should have an equal amount of samples accociated to them.

\begin{table}
\centering
\begin{tabular}{l|c|c}
    Class & Quality Range & Class ID (red/white)\\
    \hline
    \hline
    Poorquality & $[0,5]$ & (0,1) \\
    \hline
    Averagequality & $[6,6]$ & (2,3) \\
    \hline
    Highquality & $[7,10]$ & (4,5) \\
    \hline
\end{tabular}
\caption{Shows the binning range for white and red wine}
\label{tab:quality-binning}
\end{table}

\section{Training the first SOM}

After we have familiarized ourselfs with the SOMToolbox we will start to train a rather small
map. We use the settings for the ``Small Map'' shown in Table~\ref{tab:settings}. If not further
noted in the description we use the standard settings.

The following list explains the settings in more detail that are shown in Table~\ref{tab:settings}.

\begin{itemize}
    \item $\alpha$ ... the learning rate
    \item $\sigma$ ... the neighbourhood radius
    \item $s_x$ ... size of the SOM grid on the x-axis
    \item $s_y$ ... size of the SOM grid on the y-axis
    \item $i$ ... the amount of iterations
\end{itemize}

For all of our experiments we prohibit the SOM Toolbox to normalize the data because we have
already done it previously.

\begin{table}
\centering
\begin{tabular}{|l|c|c|c|c|c|c|}
    \hline
    Small Map & $\alpha = 0.7$ & $\sigma = 7$ & $s_x=5$ & $s_y=5$ & $i=1.500$ \\
    \hline
    Big Map & $\alpha = 0.7$ & $\sigma = 7$ & $s_x=50$ & $s_y=50$ & $i=1.500$ \\
    \hline
    Middle sized Map & $\alpha = 0.7$ & $\sigma = 7$ & $s_x=18$ & $s_y=18$ & $i=1.500$ \\
    \hline
\end{tabular}
\caption{Settings for all of our experiments}
\label{tab:settings}
\end{table}

\subsection{Small SOM}

We consider a $5\times5$ grid for our first experiment very small. WQ has more than 6000 entries.
This will result in many input samples grouped together to very few nodes. Our first trained
SOM is shown in Figure~\ref{fig:wine-small-hit-histogram} using the hit histogram visualization.
It can reveal dense areas on the map and in our opinion shows that the map size is too small.
We only have few units that are not fully colored in red. This is kind of a vertical line separating
the first 4 vertical units on the top left from the rest of the map. We conclude that there are
too many input samples are mapped to every unit int the SOM.

\begin{figure}
\centering
\includegraphics[width=0.5\linewidth]{img/wine-small-hit-histogram}
\caption{Hit histogram of a $5\times5$ SOM Map}
\label{fig:wine-small-hit-histogram}
\end{figure}

Another visualization hint for dense areas is the P-Matrix. It is shown in Figure~\ref{fig:wine-small-p-matrix} using
the default settings. We see that there are dense areas to the bottom right of the map and there are lesser
dense areas to the top right of the map.
By increasing the P-radius from 3 to 5 the map nearly turns into one big red pixel.
Comparing it to Figure~\ref{fig:wine-small-hit-histogram} it shows that there are many input samples densly packed
onto the lower right grid of the map. When we later increase the SOM size, this data might spread out. We
will also evaluate later to make the map rectangular because the units seem to spread into on direction.

\begin{figure}
\centering
\includegraphics[width=0.5\linewidth]{img/wine-small-p-matrix}
\caption{P-Matrix Visualization of a $5\times5$ SOM Map}
\label{fig:wine-small-p-matrix}
\end{figure}

We also present the Activity Histogram of our small SOM in Figure~\ref{fig:wine-small-activity-histogram}.
Interesting is the blue part on the mid left that is separated by a slight green/yellow.
On the bottom right we again find a very dense area because of the grayish/yellow color. This this matches
well with the P-Matrix visualization. Judging from the Activity Histogram there are 2 very distinct clusters
and maybe one cluster that is very intermixed (green/yellow). This could either be that the
red and white wine are very distinct or two subranges of the quality are dominated by ranges of the input.

\begin{figure}
\centering
\includegraphics[width=0.5\linewidth]{img/wine-small-activity-histogram}
\caption{Activity histogram of a $5\times5$ SOM Map}
\label{fig:wine-small-activity-histogram}
\end{figure}

\subsection{Quality measurements}

In the following we take a look at some quality measures available in SOMToolbox.
If not noted otherwise the default parameter values are used for all quality measures.
First we compare the Quantization Error (QE) and the Mean Quantization Error (MQE) shown in
Figure~\ref{fig:wine-small-quant-error} and~\ref{fig:wine-small-mean-quant-error} respectivley.
We already noted that the lower right is a very dense area. In the MQE these units have a very low
MQE rate thus they are colored blue. In the QE Figure the values are much more scattered. In
the top left it has a very high value thus it seems that there is an area that is not that dense
and has many items with a long distance. This we also can confirm using the MQE, because the division
by the nodes does not set a low value the value in MQE. It also reveals that there is potentially even a more sparse
region as the node at position $1\times1$ in MQE shows.

\begin{figure}
\centering
\includegraphics[width=0.5\linewidth]{img/wine-small-quant-error}
\caption{Quantization error of a $5\times5$ SOM Map}
\label{fig:wine-small-quant-error}
\end{figure}

\begin{figure}
\centering
\includegraphics[width=0.5\linewidth]{img/wine-small-mean-quant-error}
\caption{Mean quantization error of a $5\times5$ SOM Map}
\label{fig:wine-small-mean-quant-error}
\end{figure}

The map is also very distorted which is shown in Figure~\ref{fig:wine-small-dist-sqrt-2}.
When comparing this to a better train map (such as the sample of Iris dataset), it is
really distorted. Given the number of values mapped to the nodes, the distortion and
high QE we conclude current SOM size is to small.

\begin{figure}
\centering
\includegraphics[width=0.5\linewidth]{img/wine-small-dist-sqrt-2}
\caption{Distoration of the map ($\sqrt{2}$) of a $5\times5$ SOM Map}
\label{fig:wine-small-dist-sqrt-2}
\end{figure}

To finish the examination of this far to small map we show the class distribution amongst the map.
At first it surprised us that the splitting is done quite evenly between red and
white wine. Giving this a second thought it is clear that they must be more easily separatable.
White and red wine have very different characteristics in sugar, acid, etc. 
White wine is colored green (light green is poor quality, green is average quality and dark green is high quality) and
red wine is colored red(light for poor quality, orange for average quality and dark for high quality).

What we also have seen in the class distribution is that white wine has far more data samples
than the red wine. Thus we apply subsampling of white wine input samples. We reduce the dataset to around 1500 red and 1500 white wine samples.
We use a random seed of 0 in our export script to repeat this experiment.

\begin{figure}
\centering
\includegraphics[width=0.5\linewidth]{img/wine-small-pie-cls}
\caption{Piechart that shows the class distribution of a $5\times5$ SOM Map}
\label{fig:wine-small-pie-cls}
\end{figure}

\section{A very big SOM}

Now we would like to examine the data in a very big SOM. The settings are displayed
in~\ref{tab:settings}. Our assumption is that it is too big if we have less than 3
samples per SOM unit.

To find out if the SOM is too big we take a look at the QE (Figure ~\ref{fig:wine-big-quant-error}) and MQE (Figure~\ref{fig:wine-big-mean-quant-error}) quality measurements.
The first thing we noticed is the huge number of interpolating/empty units on the map. By comparing QE and MQE we noticed that there is not
a significant differance in the color coding. This is caused by fact that map is too big and only a small number of samples
are mapped to the units.
At this point we immediatly stopped to evaluate this huge SOM and decreased the size to $30\times30$. The main reason was that we had problems
with the SOM Toolbox to display various visualizations. Even the $30\times30$ SOM confirms our assumption that big sized SOMs have
many interpolating units. The picture of the QE and MQE are quite the same but only with fewer nodes.


\begin{figure}
\centering
\includegraphics[width=\linewidth]{img/wine-big-quant-error}
\caption{Quantisation error of a $50\times50$ SOM Map}
\label{fig:wine-big-quant-error}
\end{figure}

\begin{figure}
\centering
\includegraphics[width=\linewidth]{img/wine-big-mean-quant-error}
\caption{Quantisation error of a $50\times50$ SOM Map}
\label{fig:wine-big-mean-quant-error}
\end{figure}

We assume that the hit histogram of the $50\times50$ SOM should not have alot of red spots.
Using the hit histogram of the smaller big SOM ($30\times30$) shown in Figure~\ref{fig:wine-big-hit-histogram} we notice that very light
colors which indicates that few or none input samples are mapped to the SOM nodes.

\begin{figure}
\centering
\includegraphics[width=\linewidth]{img/wine-big-hit-histogram}
\caption{Hit histogram of a $30\times30$ SOM Map}
\label{fig:wine-big-hit-histogram}
\end{figure}

In Figure~\ref{wine-big-p-matrix} we se a huge red spot in the center of the map. This indicates that
the data is very dense at this area. Comparing it to the small SOM P-Matrix there is this dense area
in the center. The difference is that the dense area is to the center right not to the bottom right as assumed with
the small one.

\begin{figure}
\centering
\includegraphics[width=\linewidth]{img/wine-big-p-matrix}
\caption{P matrix visualization of a $30\times30$ SOM Map}
\label{fig:wine-big-p-matrix}
\end{figure}

The last quality measure we apply for this map is the topographic product.
We used a setting of $k=1$. The visualisation reveals a distortion in the
SOM borders and a horizontal line of distortion in the center of the map.
When increasing the number of neighbours ($k$) we still see the distortion
we mentioned earlier.

\begin{figure}
\centering
\includegraphics[width=\linewidth]{img/wine-big-topo-product}
\caption{Topographic product of the $30\times30$ SOM Map}
\label{fig:wine-big-topo-product}
\end{figure}

In the Figure~\ref{fig:wine-big-u-matrix} we show the U-Matrix visualization.
It reveals the possible cluster bounderies. We have many coherent regions/valleys located in
the upper and lower parts of the map. They are separated by a horizontal mountain regions
or high values in the center which depict the possible cluster boundaries.

\begin{figure}
\centering
\includegraphics[width=\linewidth]{img/wine-big-u-matrix}
\caption{U Matrix of the $30\times30$ SOM Map}
\label{fig:wine-big-u-matrix}
\end{figure}

The potential cluster boundaries displayed in the SDH visualization in Figure~\ref{fig:wine-big-smoothed-data-histogram}.

\begin{figure}
\centering
\includegraphics[width=\linewidth]{img/wine-big-smoothed-data-histogram}
\caption{Smoothed data histograms of the $30\times30$ SOM Map}
\label{fig:wine-big-smoothed-data-histogram}
\end{figure}

In addition the activity historygram in Figure~\ref{fig:wine-big-activity-histogram} shows possible
topology violations. There is not a gradient of the weight vectors measured in distance. In can
be seen in the upper cluster where blue area is separated by green.

\begin{figure}
\centering
\includegraphics[width=\linewidth]{img/wine-big-activity-histogram}
\caption{Activity histogram of the $30\times30$ SOM Map}
\label{fig:wine-big-activity-histogram}
\end{figure}

The analysed visulations/quality measures depict that we SOM is oversized.
In the following we will try to approximate a better size.

\section{Normal SOM}

In the following we show a middle sized map. Parameters are included in the Table~\ref{tab:settings}.
On the first view of QE in Figure~\ref{fig:wine-mid-quant-error} and MQE~\ref{fig:wine-mid-quant-error} we see
that there are three interpolating units. Comparing the two we notice that the distant data samples are
mapped to the units located along the a horzontal line in the center or the potential cluster boundaries.
The U-Matrix in Figure~\ref{fig:wine-mid-u-matrix} reveals the cluster bounderies shown with high mountain values
along the line mentioned earlier.

Again we used the SDH visualization. In the earlier settings we used a smoothing factor of 149. We decreased it to
115 and managed to separate the two main clusters shown in Figure~\ref{fig:wine-mid-soothed-data-histogram}.

There are potential topology violations located in the upper cluster. We see this in Figure~\ref{fig:wine-mid-activity-histogram}.
Like in the big SOM there are three blue areas separated by green an yellow. By observing the neighbourhood graph
shown in Figure~\ref{fig:wine-mid-radius-neighbourhood-grah}. For the radius was set to 0.4. We notice three red lines connecting the blue areas.


\begin{figure}
\centering
\includegraphics[width=\linewidth]{img/wine-mid-quant-error}
\caption{Quantisation error of a $18\times18$ SOM Map}
\label{fig:wine-mid-quant-error}
\end{figure}

\begin{figure}
\centering
\includegraphics[width=\linewidth]{img/wine-mid-mean-quant-error}
\caption{Quantisation error of a $18\times18$ SOM Map}
\label{fig:wine-mid-mean-quant-error}
\end{figure}

\begin{figure}
\centering
\includegraphics[width=\linewidth]{img/wine-mid-p-matrix}
\caption{P matrix visualization of a $18\times18$ SOM Map}
\label{fig:wine-mid-p-matrix}
\end{figure}

\begin{figure}
\centering
\includegraphics[width=\linewidth]{img/wine-mid-radius-neighbourhood-graph}
\caption{Topographic product of the $18\times18$ SOM Map}
\label{fig:wine-mid-radius-neighbourhood-graph}
\end{figure}

\begin{figure}
\centering
\includegraphics[width=\linewidth]{img/wine-mid-u-matrix}
\caption{U Matrix of the $18\times18$ SOM Map}
\label{fig:wine-mid-u-matrix}
\end{figure}

\begin{figure}
\centering
\includegraphics[width=\linewidth]{img/wine-mid-smoothed-data-histogram}
\caption{Smoothed data histograms of the $18\times18$ SOM Map}
\label{fig:wine-mid-smoothed-data-histogram}
\end{figure}

\begin{figure}
\centering
\includegraphics[width=\linewidth]{img/wine-mid-activity-histogram}
\caption{Activity histogram of the $18\times18$ SOM Map}
\label{fig:wine-mid-activity-histogram}
\end{figure}






\bibliography{ref}
\bibliographystyle{plain}



\end{document}
